% Listing style definition for the Lean Theorem Prover.
% Defined by Jeremy Avigad, 2015, by modifying Assia Mahboubi's SSR style.
% Unicode replacements taken from Olivier Verdier's unixode.sty

\lstdefinelanguage{infinity} {

% Anything betweeen $ becomes LaTeX math mode
mathescape=true,
% Comments may or not include Latex commands
texcl=false,

% keywords, list taken from lean-syntax.el
morekeywords=[1]{
record,data,inductive,extend,enum,theorem,
Type,Path,unit,Unit,Nat,List,let,in,eq,type,sh,sub,star,
var,dep,norm,fun,app,lambda,arrow,pi,case,receive,spawn,
send,name,list,nat,tele,case,id,sigma,pair,fst,snd,id,idPair,idJ,
branch,label,ctor,sum,prod,ty,lam,split,Glue,glue,unglue,Pi,Equ,
Sigma,pr1,pr2,Beta,Eta,Beta1,Beta2,Eta2,refl,ap,apply,apd,J,
process,execute,storage,axiom,undefined,comp,PathP,fill,
precategory,catfunctor,equiv,trans,base,loop,S1,S2,pub,
struct,Vec,Arc,Cell,sub,rcv,snd,spawn,module,import,
coerce,cong,Fixpoint,Parameter,Definition,CoInductive,
Require,Import,CoFixpoint,remote,ac,total,fix,ext,fn,
call,true,false,is_defined,datum,id_intro,id_elim,
where,bool,maybe,either,unit,empty,list,stream,vector,fin,
def,definition,axiom,module,where,let,in,case,split,data,record,form,elim,
import, prelude, protected, private, noncomputable, definition, meta, renaming,
hiding, exposing, parameter, parameters, begin, conjecture, constant, constants,
hypothesis, lemma, corollary, variable, variables, premise, premises, theory,
print, theorem, proposition, example, abstract,
open, as, export, override, axiom, axioms, inductive, with, without,
structure, record, universe, universes,
alias, help, precedence, reserve, declare_trace, add_key_equivalence,
match, infix, infixl, infixr, notation, postfix, prefix,
eval, vm_eval, check, coercion, end, this, suppose,
using, using_well_founded, namespace, section, fields,
attribute, local, set_option, extends, include, omit, classes, class,
instances, coercions, attributes, raw, replacing,
calc, have, show, suffices, by, in, at, let, forall, Pi, fun,
exists, if, dif, then, else, assume, take, obtain, from, aliases, register_simp_ext,
mutual, def, run_command
},

% Sorts
morekeywords=[2]{Type, Prop, Type*, Type₀, Type₁, Type₂, Type₃},

% tactics, list taken from lean-syntax.el
% morekeywords=[3]{
% Cond, or_else, then, try, when, assumption, eassumption, rapply,
% apply, fapply, eapply, rename, intro, intros, all_goals, fold, focus, focus_at,
% generalize, generalizes, clear, clears, revert, reverts, back, beta, done, exact, rexact,
% refine, repeat, whnf, rotate, rotate_left, rotate_right, inversion, cases, rewrite,
% xrewrite, krewrite, blast, simp, esimp, unfold, change, check_expr, contradiction,
% exfalso, split, existsi, constructor, fconstructor, left, right, injection, congruence, reflexivity,
% symmetry, transitivity, state, induction, induction_using, fail, append,
% substvars, now, with_options, with_attributes, with_attrs, note
% },

% modifiers, taken from lean-syntax.el
% note: 'otherkeywords' is needed because these use a different symbol.
% this command doesn't allow us to specify a number -- they are put with [1]
% otherkeywords={
% [persistent], [notation], [visible], [instance], [trans_instance],
% [class], [parsing-only], [coercion], [unfold_full], [constructor],
% [reducible], [irreducible], [semireducible], [quasireducible], [wf],
% [whnf], [multiple_instances], [none], [decl], [declaration],
% [relation], [symm], [subst], [refl], [trans], [simp], [congr], [unify],
% [backward], [forward], [no_pattern], [begin_end], [tactic], [abbreviation],
% [reducible], [unfold], [alias], [eqv], [intro], [intro!], [elim], [grinder],
% [localrefinfo], [recursor]
% },

% Various symbols
literate=
{≔}{{\ensuremath{\mathrm{:=}}}}1
{α}{{\ensuremath{\mathrm{\alpha}}}}1
{β}{{\ensuremath{\mathrm{\beta}}}}1
{γ}{{\ensuremath{\mathrm{\gamma}}}}1
{δ}{{\ensuremath{\mathrm{\delta}}}}1
{ε}{{\ensuremath{\mathrm{\varepsilon}}}}1
{ζ}{{\ensuremath{\mathrm{\zeta}}}}1
{η}{{\ensuremath{\mathrm{\eta}}}}1
{θ}{{\ensuremath{\mathrm{\theta}}}}1
{ι}{{\ensuremath{\mathrm{\iota}}}}1
{κ}{{\ensuremath{\mathrm{\kappa}}}}1
{μ}{{\ensuremath{\mathrm{\mu}}}}1
{ν}{{\ensuremath{\mathrm{\nu}}}}1
{ξ}{{\ensuremath{\mathrm{\xi}}}}1
{π}{{\ensuremath{\mathrm{\mathnormal{\pi}}}}}1
{ρ}{{\ensuremath{\mathrm{\rho}}}}1
{σ}{{\ensuremath{\mathrm{\sigma}}}}1
{τ}{{\ensuremath{\mathrm{\tau}}}}1
{φ}{{\ensuremath{\mathrm{\varphi}}}}1
{χ}{{\ensuremath{\mathrm{\chi}}}}1
{ψ}{{\ensuremath{\mathrm{\psi}}}}1
{ω}{{\ensuremath{\mathrm{\omega}}}}1
{Π}{{\ensuremath{\mathrm{\Pi}}}}1
{Γ}{{\ensuremath{\mathrm{\Gamma}}}}1
{Δ}{{\ensuremath{\mathrm{\Delta}}}}1
{Θ}{{\ensuremath{\mathrm{\Theta}}}}1
{Λ}{{\ensuremath{\mathrm{\Lambda}}}}1
{Σ}{{\ensuremath{\mathrm{\Sigma}}}}1
{Φ}{{\ensuremath{\mathrm{\Phi}}}}1
{Ξ}{{\ensuremath{\mathrm{\Xi}}}}1
{Ψ}{{\ensuremath{\mathrm{\Psi}}}}1
{Ω}{{\ensuremath{\mathrm{\Omega}}}}1
{ℵ}{{\ensuremath{\aleph}}}1
{≤}{{\ensuremath{\leq}}}1
{≥}{{\ensuremath{\geq}}}1
{≠}{{\ensuremath{\neq}}}1
{≈}{{\ensuremath{\approx}}}1
{≡}{{\ensuremath{\equiv}}}1
{≃}{{\ensuremath{\simeq}}}1
{≤}{{\ensuremath{\leq}}}1
{≥}{{\ensuremath{\geq}}}1
{∂}{{\ensuremath{\partial}}}1
{∆}{{\ensuremath{\triangle}}}1 % or \laplace?
{∫}{{\ensuremath{\int}}}1
{∑}{{\ensuremath{\mathrm{\Sigma}}}}1
{→}{{\ensuremath{\rightarrow}}}1
{⊥}{{\ensuremath{\perp}}}1
{∞}{{\ensuremath{\infty}}}1
{∂}{{\ensuremath{\partial}}}1
{∓}{{\ensuremath{\mp}}}1
{±}{{\ensuremath{\pm}}}1
{×}{{\ensuremath{\times}}}1
{⊕}{{\ensuremath{\oplus}}}1
{⊗}{{\ensuremath{\otimes}}}1
{⊞}{{\ensuremath{\boxplus}}}1
{∇}{{\ensuremath{\nabla}}}1
{√}{{\ensuremath{\sqrt}}}1
{⬝}{{\ensuremath{\cdot}}}1
{•}{{\ensuremath{\cdot}}}1
{∘}{{\ensuremath{\circ}}}1
{⁻}{{\ensuremath{^{-}}}}1
{▸}{{\ensuremath{\blacktriangleright}}}1
{∧}{{\ensuremath{\wedge}}}1
{∨}{{\ensuremath{\vee}}}1
{¬}{{\ensuremath{\neg}}}1
{⊢}{{\ensuremath{\vdash}}}1
{⟨}{{\ensuremath{\langle}}}1
{⟩}{{\ensuremath{\rangle}}}1
{↦}{{\ensuremath{\mapsto}}}1
{→}{{\ensuremath{\rightarrow}}}1
{↔}{{\ensuremath{\leftrightarrow}}}1
{⇒}{{\ensuremath{\Rightarrow}}}1
{⟹}{{\ensuremath{\Longrightarrow}}}1
{⇐}{{\ensuremath{\Leftarrow}}}1
{⟸}{{\ensuremath{\Longleftarrow}}}1
{∩}{{\ensuremath{\cap}}}1
{∪}{{\ensuremath{\cup}}}1
{⊂}{{\ensuremath{\subseteq}}}1
{⊆}{{\ensuremath{\subseteq}}}1
{⊄}{{\ensuremath{\nsubseteq}}}1
{⊈}{{\ensuremath{\nsubseteq}}}1
{⊃}{{\ensuremath{\supseteq}}}1
{⊇}{{\ensuremath{\supseteq}}}1
{⊅}{{\ensuremath{\nsupseteq}}}1
{⊉}{{\ensuremath{\nsupseteq}}}1
{∈}{{\ensuremath{\in}}}1
{∉}{{\ensuremath{\notin}}}1
{∋}{{\ensuremath{\ni}}}1
{∌}{{\ensuremath{\notni}}}1
{∅}{{\ensuremath{\emptyset}}}1
{∖}{{\ensuremath{\setminus}}}1
{†}{{\ensuremath{\dag}}}1
{ℕ}{{\ensuremath{\mathbb{N}}}}1
{ℤ}{{\ensuremath{\mathbb{Z}}}}1
{ℝ}{{\ensuremath{\mathbb{R}}}}1
{ℚ}{{\ensuremath{\mathbb{Q}}}}1
{ℂ}{{\ensuremath{\mathbb{C}}}}1
{⌞}{{\ensuremath{\llcorner}}}1
{⌟}{{\ensuremath{\lrcorner}}}1
{⦃}{{\ensuremath{ \{\!| }}}1
{⦄}{{\ensuremath{ |\!\} }}}1
{₁}{{\ensuremath{_1}}}1
{₂}{{\ensuremath{_2}}}1
{₃}{{\ensuremath{_3}}}1
{₄}{{\ensuremath{_4}}}1
{₅}{{\ensuremath{_5}}}1
{₆}{{\ensuremath{_6}}}1
{₇}{{\ensuremath{_7}}}1
{₈}{{\ensuremath{_8}}}1
{₉}{{\ensuremath{_9}}}1
{₀}{{\ensuremath{_0}}}1
{¹}{{\ensuremath{^1}}}1
{ₙ}{{\ensuremath{_n}}}1
{ₘ}{{\ensuremath{_m}}}1
{↑}{{\ensuremath{\uparrow}}}1
{↓}{{\ensuremath{\downarrow}}}1
{▸}{{\ensuremath{\triangleright}}}1
{Σ}{{\color{symbolcolor}\ensuremath{\Sigma}}}1
{Π}{{\color{symbolcolor}\ensuremath{\Pi}}}1
{∀}{{\color{symbolcolor}\ensuremath{\forall}}}1
{∃}{{\color{symbolcolor}\ensuremath{\exists}}}1
{λ}{{\color{symbolcolor}\ensuremath{\mathrm{\lambda}}}}1
{=}{{\color{symbolcolor}=}}1
{<}{{\color{symbolcolor}<}}1
{(}{(}1
{)}{)}1
{+}{{\color{symbolcolor}+}}1
{*}{{\color{symbolcolor}*}}1,

% Comments
%comment=[s][\itshape \color{commentcolor}]{/-}{-/},
morecomment=[s][\color{commentcolor}]{/-}{-/},
morecomment=[l][\itshape \color{commentcolor}]{--},

% Spaces are not displayed as a special character
showstringspaces=false,

% keep spaces
keepspaces=true,

% String delimiters
morestring=[b]",
morestring=[d],

% Size of tabulations
tabsize=3,

% Enables ASCII chars 128 to 255
extendedchars=true,

% Case sensitivity
sensitive=true,

% Automatic breaking of long lines
breaklines=true,

% Default style fors listingsred
basicstyle=\bf\ttfamily\footnotesize,

% Position of captions is bottom
captionpos=b,

% Full flexible columns
columns=[l]fullflexible,


% Style for (listings') identifiers
identifierstyle={\ttfamily\color{black}},
% Note : highlighting of Coq identifiers is done through a new
% delimiter definition through an lstset at the begining of the
% document. Don't know how to do better.

% Style for declaration keywords
keywordstyle=[1]{\ttfamily\color{keywordcolor}},

% Style for sorts
keywordstyle=[2]{\ttfamily\color{sortcolor}},

% Style for tactics keywords
% keywordstyle=[3]{\ttfamily\color{tacticcolor}},

% Style for attributes
% keywordstyle=[4]{\ttfamily\color{attributecolor}},

% Style for strings
stringstyle=\ttfamily,

% Style for comments
% commentstyle={\ttfamily\footnotesize },

}

